\documentclass[12pt,class=article,crop=false,preview=false]{standalone}

\usepackage[utf8]{inputenc}

\usepackage{amsmath}
\usepackage{amssymb}
\usepackage{siunitx}
\sisetup{range-phrase=--}
\sisetup{range-units=single}
\sisetup{per-mode=symbol}

\usepackage{xcolor}
\usepackage[unicode]{hyperref}
\usepackage{minted}
\usepackage{fullpage}

\usepackage{tikz}
\usepackage[siunitx]{circuitikz}
\usepackage{tikz-3dplot}

\tdplotsetmaincoords{110}{-12}
\usetikzlibrary {arrows.meta}
\usepackage{stanli}

\usepackage{enumitem}

\newcommand{\mat}[1]{\left[\begin{matrix}#1\end{matrix}\right]}
\newcommand{\vectwo}[2]{\left[\begin{matrix}#1\\#2\end{matrix}\right]}
\newcommand{\mattwo}[4]{\left[\begin{matrix}#1 & #3\\#2 & #4\end{matrix}\right]}

\DeclareMathOperator{\dive}{div}
\DeclareMathOperator{\grad}{grad}
\DeclareMathOperator{\tr}{tr}

\newcommand{\rr}[2]{\frac{\partial #1}{\partial #2}}
\newcommand{\pr}[1]{\frac{\partial}{\partial #1}}
\newtheorem{exercise}{Exercise}[section]

\usepackage{comment}
\specialcomment{solution}{{\bf Solution}\quad}{}

\setcounter{section}{3}

\excludecomment{solution}

\begin{document}
\subsection*{Exercises}

\begin{exercise}
    Show that stress tensor $\sigma$ is symmetric, by assuming $\sigma$ is constant in a region and calculating the moment acting on a small rectangular prism.
\end{exercise}

\begin{solution}
    Let $\sigma = \mat{
        \sigma_{xx}&\sigma_{xy}&\sigma_{xz}\\
        \sigma_{yx}&\sigma_{yy}&\sigma_{yz}\\
        \sigma_{zx}&\sigma_{zy}&\sigma_{zz}}$ be constant. Let us have a small rectangular prism $[-a,a]\times[-a,a]\times[-a,a]$, and let us name its six faces as A,B,C,D,E and F:
    
    \begin{tikzpicture}[tdplot_main_coords]
        \draw[->] (-2,0,0) -- (2,0,0);
        \draw[->] (0,-2,0) -- (0,2,0);
        \draw[->] (0,0,-2) -- (0,0,2);
        \node[anchor=south east] at (2,0,0) {x};
        \node[anchor=south east] at (0,2,0) {y};
        \node[anchor=north east] at (0,0,2) {z};
        \draw (1,-1,-1) -- (1,-1,1) -- (-1,-1,1) -- (-1,1,1) -- (-1,1,-1) -- (1,1,-1) -- cycle;
        \draw (1,-1,1) -- (1,1,1) -- (-1,1,1) (1,1,-1) -- (1,1,1);
        \draw[dashed] (1,-1,-1) -- (-1,-1,-1) -- (-1,-1,1) (-1,1,-1) -- (-1,-1,-1);
        \begin{scope}[shift={(5,0,0)}]
        \draw (1,-1,-1) -- (1,-1,1) -- (-1,-1,1) -- (-1,1,1) -- (-1,1,-1) -- (1,1,-1) -- cycle;
        \draw (1,-1,1) -- (1,1,1) -- (-1,1,1) (1,1,-1) -- (1,1,1);
        \draw[dashed] (1,-1,-1) -- (-1,-1,-1) -- (-1,-1,1) (-1,1,-1) -- (-1,-1,-1);
        \node at (1,0,0) {A};
        \node[color=gray] at (-1,0,0) {B};
        \node at (0,1,0) {C};
        \node[color=gray] at (0,-1,0) {D};
        \node at (0,0,1) {E};
        \node[color=gray] at (0,0,-1) {F};
        \end{scope}
        \begin{scope}[shift={(10,0,0)}]
        \draw (1,-1,-1) -- (1,-1,1) -- (-1,-1,1) -- (-1,1,1) -- (-1,1,-1) -- (1,1,-1) -- cycle;
        \draw (1,-1,1) -- (1,1,1) -- (-1,1,1) (1,1,-1) -- (1,1,1);
        \draw[dashed] (1,-1,-1) -- (-1,-1,-1) -- (-1,-1,1) (-1,1,-1) -- (-1,-1,-1);
        \draw[->] (0,0,-2) -- (0,0,2);
        \node[anchor=north east] at (0,0,2) {z};
        \draw[-Latex,very thick] (1,-0.9,0) -- (1,0.9,0);
        \draw[-Latex,very thick] (-0.9,1,0) -- (0.9,1,0);
        \draw[-Latex,very thick,gray] (-1,0.9,0) -- (-1,-0.9,0);
        \draw[-Latex,very thick,gray] (0.9,-1,0) -- (-0.9,-1,0);
        \node[anchor=west] at (1,0,0) {$\sigma_{yx}$};
        \node[anchor=north east] at (0,1,0) {$\sigma_{xy}$};
        \end{scope}
    \end{tikzpicture}
    
    The normal direction to A is $[1,0,0]$ and force acting on this face is $\sigma n = \mat{\sigma_{xx}\\\sigma_{yx}\\\sigma_{zx}}$. Moment of this force around the $z$ axis is $-a\sigma_{yx}$ times the area $a^2$. Calculating the same for $B$ we get again $-a^3\sigma_{yx}$. Calculating it for $C$ and $D$ we get $a^3\sigma_{xy}$, whereas for $E$ and $F$ we get $0$. For the moment to be zero we have to have:
    \[2a^3(-\sigma_{yx} + \sigma_{xy}) = 0\]
    Thus, $\sigma_{yx} = \sigma_{xy}$. Calculating moments around $y$ and $x$ axis we will get $\sigma_{zx} = \sigma_{xz}$ and $\sigma_{zy} = \sigma_{yz}$.

    (option 2): Moment of force is the cross product of force $f$ and arm $r$ vectors. In our case $r$ is a vector from center to a point on the surface ($r=[x,y,z]$), and $f=\sigma n$. We can express is as an integral:
    
    \[\int_{\partial\Omega} r\times f = \int_{\partial\Omega} r\times (\sigma n) = \int_{\partial\Omega}\Gamma^{ijk}r_j\sigma_{kl} n_l= \int_{\Omega}\pr{x_l}\Gamma^{ijk}r_j\sigma_{kl}= \int_{\Omega}\Gamma^{ijk}\delta_{lj}\sigma_{kl}= \int_{\Omega}\Gamma^{ijk}\sigma_{kj}\]
\end{solution}

\begin{exercise}
Let us have a axially-loaded member with dimensions $L\times a\times a$, with cross-section $a^2$ and length $L$, which is fixed at one end and loaded with force $F$ on the other.
\begin{enumerate}[label=\alph*)]
    \item By how much will this bar elongate?
    \item By how much will this bar shrink in other two directions?
    \item Can you write out the displacement field $u(x,y,z)=[u_x,u_y,u_z]$, assuming that the beam is fixed in $[0,0,0]$, and the beam is oriented along the $x$ direction? Hint: it will have a form of $u(x,y,z)=[\alpha x,\beta y,\gamma z]$.
    \item Calculate the strain $\varepsilon_{ij} = \frac{1}{2}\left(\rr{u_i}{x_j}+\rr{u_j}{x_i}\right)$
    \item Write out the stress tensor.
    \item Show that $\lambda = \frac{E\nu}{(1+\nu)(1-2\nu)}$ and $\mu=\frac{E}{2(1+\nu)}$, by substituting the stress and strain into the constitutive equation $\sigma = 2\mu\varepsilon + \lambda I \tr{\varepsilon}$.
\end{enumerate}
\end{exercise}

\begin{solution}
For a axially-loaded bar we have $E\frac{d_L}{L} = \frac{F}{A}$, where $d_L$ is the displacement in the axial direction. In the other two directions, by the definition of Poisson ratio, $\nu$, we have $\frac{d_a}{a}=\nu\frac{d_L}{L}$.

If we assume the beam to be fixed in $(0,0,0)$ we can write out the deformation field as:
\[u(x,y,z)=\frac{F}{AE}[x,-\nu y,-\nu z]\]
From this we can calculate the strain:
\[\varepsilon=\frac{F}{AE}\mat{1&0&0\\0&-\nu&0\\0&0&-\nu}\]
The stress tensor has only $xx$ component and is:
\[\sigma=\mat{F/A&0&0\\0&0&0\\0&0&0}\]
Substituting them into the constitutive equation:
\begin{align*}
\mat{F/A&0&0\\0&0&0\\0&0&0} &= 2\mu\frac{F}{AE}\mat{1&0&0\\0&-\nu&0\\0&0&-\nu} + \lambda I \tr{\frac{F}{AE}\mat{1&0&0\\0&-\nu&0\\0&0&-\nu}} \\
\mat{F/A&0&0\\0&0&0\\0&0&0} &= 2\mu\frac{F}{AE}\mat{1&0&0\\0&-\nu&0\\0&0&-\nu} + \lambda \mat{1&0&0\\0&1&0\\0&0&1} \frac{F}{AE}(1-2\nu) \\
E\mat{1&0&0\\0&0&0\\0&0&0} &= 2\mu\mat{1&0&0\\0&-\nu&0\\0&0&-\nu} + \lambda \mat{1&0&0\\0&1&0\\0&0&1}(1-2\nu) 
\end{align*}
\end{solution}

\begin{exercise}
What stress regimes (which components of stress you expect to be zero) in the below tray with legs? In two situations:
\begin{enumerate}[label=\alph*)]
\item all joints being fixed (welded).
\item all joints being flexible (hinges).
\end{enumerate}

    \begin{tikzpicture}[tdplot_main_coords]
        \coordinate (a) at (-2,-2,-2);
        \coordinate (b) at (-2,2,-2);
        \coordinate (c) at (2,-2,-2);
        \coordinate (d) at (2,2,-2);
        %\point{a}{0}{0};
        \draw[very thick] (a) -- (-2,-2,1);
        \draw[fill,gray,draw=black,very thick] (-2,-2,0) -- (-2,2,0) -- (2,2,0) -- (2,-2,0) -- cycle;
        \draw[fill,gray,draw=black,very thick] (-2,-2,0) -- (-2,-2,1) -- (-2,2,1) -- (-2,2,0) -- cycle;
        \draw[fill,gray,draw=black,very thick] (-2,-2,0) -- (-2,-2,1) -- (2,-2,1) -- (2,-2,0) -- cycle;
        \draw[fill,gray,draw=black,very thick] (2,2,0) -- (2,2,1) -- (-2,2,1) -- (-2,2,0) -- cycle;
        \draw[fill,gray,draw=black,very thick] (2,2,0) -- (2,2,1) -- (2,-2,1) -- (2,-2,0) -- cycle;
        \draw[very thick] (b) -- (-2,2,1);
        \draw[very thick] (c) -- (2,-2,1);
        \draw[very thick] (d) -- (2,2,1);
        \support{1}{a}; \hinge{1}{a};
        \support{1}{b}; \hinge{1}{b};
        \support{1}{c}; \hinge{1}{c};
        \support{1}{d}; \hinge{1}{d};
    \end{tikzpicture}
\end{exercise}

\end{document}