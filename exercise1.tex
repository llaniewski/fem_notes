\documentclass[12pt,class=article,crop=false,preview=false]{standalone}

\usepackage[utf8]{inputenc}

\usepackage{amsmath}
\usepackage{amssymb}
\usepackage{siunitx}
\sisetup{range-phrase=--}
\sisetup{range-units=single}
\sisetup{per-mode=symbol}

\usepackage{xcolor}
\usepackage[unicode]{hyperref}
\usepackage{minted}
\usepackage{fullpage}

\usepackage{tikz}
\usepackage[siunitx]{circuitikz}
\usepackage{tikz-3dplot}

\tdplotsetmaincoords{110}{-12}
\usetikzlibrary {arrows.meta}
\usepackage{stanli}

\usepackage{enumitem}

\newcommand{\mat}[1]{\left[\begin{matrix}#1\end{matrix}\right]}
\newcommand{\vectwo}[2]{\left[\begin{matrix}#1\\#2\end{matrix}\right]}
\newcommand{\mattwo}[4]{\left[\begin{matrix}#1 & #3\\#2 & #4\end{matrix}\right]}

\DeclareMathOperator{\dive}{div}
\DeclareMathOperator{\grad}{grad}
\DeclareMathOperator{\tr}{tr}

\newcommand{\rr}[2]{\frac{\partial #1}{\partial #2}}
\newcommand{\pr}[1]{\frac{\partial}{\partial #1}}
\newtheorem{exercise}{Exercise}[section]

\usepackage{comment}
\specialcomment{solution}{{\bf Solution}\quad}{}

\setcounter{section}{1}
%\excludecomment{solution}

\begin{document}

\subsection*{Exercises}
These are exercises for you to practice some useful math before the next lecture. Note: exercises 1 and 3 are about deriving expressions needed, not calculating the final values. One can think of it this way: what would I need to calculate it in Python or Matlab?

\begin{exercise}
Express the equations describing the below electric circuit in a matrix form.
\begin{enumerate}[label=\alph*)]
    \item Identify the element matrices.
    \item Is the system square and non-singular? If not, how can it be adjusted?
    \item How to calculate total energy consumption of this circuit in Watts?
\end{enumerate}
\begin{center}
\newcommand{\nlab}[1]{node[label={[font=\footnotesize]-100:$#1$}] {}}
\begin{circuitikz}
\draw (0,0) \nlab{V_0} to[R=3<\kilo\ohm>,*-*] (2,0) \nlab{V_1} to[resistor, l=3<\kilo\ohm>,-*] (4,0) \nlab{V_2} to[resistor, l=3<\kilo\ohm>,-*] (6,0) \nlab{V_3}
(2,0) -- (2,1) to[resistor, l=4<\kilo\ohm>] (6,1) -- (6,0)
(6,0) -- (6,-1) to[battery, l=5<\volt>] (0,-1) -- (0,0);
\end{circuitikz}
\end{center}
\end{exercise}

\begin{solution}
Let's call the resistors $A,B,C$ and $D$. From the Ohm law:
\begin{align*}
I_A &= \frac{1}{R_A}\left(V_1 - V_0\right) & I_B &= \frac{1}{R_B}\left(V_2 - V_1\right)\\
I_C &= \frac{1}{R_C}\left(V_3 - V_2\right) & I_D &= \frac{1}{R_D}\left(V_3 - V_1\right)
\end{align*}
Let us call the battery current $I_\text{battery}$. Writing out the balance of current in each node, we get:
\begin{align*}
I_\text{battery} - I_A &= 0\\
I_A - I_B - I_D &= 0\\
I_B - I_C &= 0\\
I_C + I_D - I_\text{battery} &= 0\\
\end{align*}

Now we substitute the currents from Ohm's law.

\emph{Option 1:} We ``forget'' about the battery for a second, and treat $I_\text{battery}$ as a input. We get:
\[
\mat{
\frac{1}{R_A} & - \frac{1}{R_A} & 0 & 0\\
-\frac{1}{R_A} & \frac{1}{R_A}+\frac{1}{R_B}+\frac{1}{R_D} & -\frac{1}{R_B} & -\frac{1}{R_D}\\
0 & -\frac{1}{R_B} & \frac{1}{R_B} + \frac{1}{R_C} & -\frac{1}{R_C}\\
0 & -\frac{1}{R_D} & -\frac{1}{R_C} & \frac{1}{R_C} + \frac{1}{R_D}
}\mat{V_0\\V_1\\V_2\\V_3} = \mat{-I_\text{battery}\\0\\0\\I_\text{battery}}
\]
The matrix is singular. We replace the first and last equation with boundary conditions $V_0=5[V]$ and $V_3=0$.

\emph{Option 2:} We treat $I_\text{battery}$ as and additional variable, and difference in voltages as an additional equation. We get:
\[
\mat{
\frac{1}{R_A} & - \frac{1}{R_A} & 0 & 0 & 1\\
-\frac{1}{R_A} & \frac{1}{R_A}+\frac{1}{R_B}+\frac{1}{R_D} & -\frac{1}{R_B} & -\frac{1}{R_D} & 0\\
0 & -\frac{1}{R_B} & \frac{1}{R_B} + \frac{1}{R_C} & -\frac{1}{R_C} & 0\\
0 & -\frac{1}{R_D} & -\frac{1}{R_C} & \frac{1}{R_C} + \frac{1}{R_D} & -1\\
1 & 0 & 0 & -1 & 0
}\mat{V_0\\V_1\\V_2\\V_3\\I_\text{battery}} = \mat{0\\0\\0\\0\\5}
\]
The matrix is singular. We replace the forth equation with $V_3=0$ to establish base voltage.

\begin{enumerate}[label=\alph*)]
    \item Whatever way you do it, the element matrices can be easily spotted and are the same as for springs:
\[\mat{\frac{1}{R} & -\frac{1}{R} \\ -\frac{1}{R} & \frac{1}{R}}\]
    \item In either option the matrix needs adjustment to be non-singular. This is because voltage levels have to be established (one has to say what is "0", as voltages is a relative measure)
    \item To calculate wattage, one can either calculate $I_\text{battery}\cdot 5[V]$, or $I_A(V_1-V_0)+I_B(V_2-V_1)+I_C(V_3-V_2)+I_D(V_3-V_1)$. A good student should notice that these expressions for each resistors are the same as:
    \[I(V_1-V_0) = \frac{1}{R}(V_1-V_0)(V_1-V_0) = \mat{V_0\\V_1}^T\mat{\frac{1}{R} & -\frac{1}{R} \\ -\frac{1}{R} & \frac{1}{R}}\mat{V_0\\V_1}\]
\end{enumerate}
\end{solution}

\begin{exercise}
For a 3D domain $\Omega$, knowing that:
\[\int_{\partial\Omega} n_i f = \int_\Omega\pr{x_i} f\]
show that:
\begin{align}
\int_{\partial\Omega} v\cdot n &= \int_\Omega\dive{v}\\ 
\int_{\partial\Omega} f n &= \int_\Omega\grad{f}\\
\int_{\partial\Omega} \rr{f}{n} &= \int_\Omega\Delta f\\
\int_\Omega g\rr{f}{x_i} &= \int_{\partial\Omega} n_i fg - \int_\Omega \frac{\partial g}{\partial x_i}f\\
\int_\Omega v\cdot\grad{f} &= \int_{\partial\Omega} (v\cdot n)f - \int_\Omega f\dive{v}\\
\int_\Omega f\Delta{g} &= \int_{\partial\Omega} g\rr{f}{n} - \int_\Omega (\grad{f})\cdot(\grad{g})
\end{align}
\end{exercise}

\begin{solution}
First two expressions in this task is a simple substitution:
\begin{align*}
    (1) = \int_{\partial\Omega} v\cdot n
    &= \int_{\partial\Omega} (v_xn_x+v_yn_y+v_zn_z) = \\
    &= \int_{\partial\Omega} v_xn_x + \int_{\partial\Omega}v_yn_y + \int_{\partial\Omega}v_zn_z = \\
    &= \int_{\Omega} \pr{x}v_x + \int_{\Omega} \pr{y}v_y + \int_{\Omega} \pr{z}v_z = \\
    &= \int_{\Omega} (\pr{x}v_x + \pr{y}v_y + \pr{z}v_z) = \\
    &= \int_{\Omega} \dive v\\
    (2) = \int_{\partial\Omega} f n
    &= \int_{\partial\Omega} f \mat{n_x & n_y & n_z} = \\
    &= \mat{\int_{\partial\Omega} f n_x & \int_{\partial\Omega} f n_y & \int_{\partial\Omega} f n_z} = \\
    &= \mat{\int_{\Omega} \pr{x}f & \int_{\Omega} \pr{y}f & \int_{\Omega} \pr{z}f} = \\
    &= \int_{\Omega} \mat{\pr{x}f & \pr{y}f & \pr{z}f} = \\
    &= \int_{\Omega} \grad f
\end{align*}
Eq. (3) follows from (1), as $\int_{\partial\Omega} \rr{f}{n} = \int_{\partial\Omega} \grad{f}\cdot n = \int_{\Omega} \dive\grad{f} = \int_{\Omega} \Delta f$. Finally the (4), (5) and (6) is just integration by parts, so they use the fact that $(fg)' = f'g + fg'$, from which it follows that $f'g = (fg)' - fg'$.
\begin{align*}
(4) = \int_\Omega g\rr{f}{x_i}
&= \int_\Omega \left(\pr{x_i}(fg) - \rr{g}{x_i}f\right) =\\
&= \int_\Omega \pr{x_i}(fg) - \int_\Omega \rr{g}{x_i}f =\\
&= \int_{\partial\Omega} n_i fg - \int_\Omega \rr{g}{x_i}f\\
(5) = \int_\Omega v\cdot\grad{f}
&= \int_\Omega \left(\dive(vf) - (\dive v)f\right) =\\
&= \int_\Omega\dive(vf) - \int_\Omega(\dive v)f =\\
&= \int_{\partial\Omega}n\cdot vf - \int_\Omega(\dive v)f\\
(6) = \int_\Omega f\Delta{g}
&= \int_\Omega \left(\dive(f\grad{g} - \grad{f}\cdot\grad{g}\right) =\\
&= \int_\Omega\dive(f\grad{g} - \int_\Omega \grad{f}\cdot\grad{g} =\\
&= \int_{\partial\Omega}fn\cdot\grad{g} - \int_\Omega \grad{f}\cdot\grad{g} =\\
&= \int_{\partial\Omega}f\rr{g}{n} - \int_\Omega \grad{f}\cdot\grad{g}\\
\end{align*}
One does not need to know or remember any of the expressions with $\grad$ or $\dive$, as the same things can be done with writing everything explicitly over $x$, $y$ and $z$ if one prefers.
\end{solution}

\begin{exercise}
    Three points, $p^0=\mat{p^0_x& p^0_y}$, $p^1=\mat{p^1_x& p^1_y}$, and $p^2=\mat{p^2_x& p^2_y}$, exist in a 2D plane. There exists a linear function, $f$, with the value of $1$ in $p^0$ and $0$ in the other two points, meaning: $f(p^0)=1$, $f(p^1)=0$ and $f(p^2)=0$. Do the following:
    \begin{enumerate}[label=\alph*)]
        \item Write out the general form of a linear function in 2D, and for the information provided write the equations its coefficients have to fulfil
        \item Express these equations as a matrix equation
        \item Express the derivatives of $f$ with respect to $x$ and $y$, in terms of these coefficients,
        \item Can you calculate the integral of $f$ over the triangle formed by the three points, without calculating these coefficients explicitly?
        \item Can you calculate the integral of the gradient of $f$ over the same triangle?
    \end{enumerate}
\end{exercise}

\begin{solution}
\begin{enumerate}[label=\alph*)]
\item Let us have a linear function $f(x,y) = A+Bx+Cy$. The three conditions can be expressed as:
\begin{align*}
    A+Bp^0_x+Cp^0_y &= 1\\
    A+Bp^1_x+Cp^1_y &= 0\\
    A+Bp^2_x+Cp^2_y &= 0
\end{align*}
\item which in matrix form is:
\[\mat{
1 & p^0_x & p^0_y\\
1 & p^1_x & p^1_y\\
1 & p^2_x & p^2_y
}\mat{A\\B\\C} = \mat{1\\0\\0}\]
\end{enumerate}

This is enough to calculete the function in a computer and we don't need to solve this equation.

But we can, of course. Subtracting second from first equation, and third from first, we get:
\begin{align*}
    B(p^0_x-p^1_x)+C(p^0_y-p^1_y) &= 1\\
    B(p^0_x-p^2_x)+C(p^0_y-p^2_y) &= 1
\end{align*}
From which we can get (e.g. by Cramer's rule):
\begin{align*}
    B &= \frac{1}{d}((p^0_y-p^2_y) - (p^0_y-p^1_y))\\
    C &= \frac{1}{d}((p^0_x-p^1_x) - (p^0_x-p^2_x))
\end{align*}
where $d = (p^0_x-p^1_x)(p^0_y-p^2_y) - (p^0_y-p^1_y)(p^0_x-p^2_x)$

\begin{enumerate}[label=\alph*)]\setcounter{enumi}{2}
    \item The derivatives are of course $B$ and $C$.
    \item The integral over the triangle can be computed analytically, but it's a pain. But there are two simple solutions. One is to notice that it's just a volume of a tetrahedron with height $1$ and the triangle as the base (which area is $\frac{d}{2}$). The other is to change variables. Here the determinant of the Jacobian will be $d$ and the integral over a simple triangle will be $\frac{1}{6}$. Either way the value of the integral is $\frac{d}{6}$
    \item The gradient is constant across the triangle, so it's integral is its value times the area, which is again $\frac{d}{2}$.
\end{enumerate}
\end{solution}

\end{document}